\documentclass[11pt,a4paper]{moderncv}
\moderncvtheme[blue]{classic}
\usepackage[utf8]{inputenc}
\usepackage[scale=0.8]{geometry}
\setlength{\hintscolumnwidth}{3cm}						% if you want to change the width of the column with the dates
%\AtBeginDocument{\setlength{\maketitlenamewidth}{6cm}}  % only for the classic theme, if you want to change the width of your name placeholder (to leave more space for your address details
%\AtBeginDocument{\recomputelengths}                     % required when changes are made to page layout lengths

% personal data
\firstname{Fernando~J.}
\familyname{Pereda}
%\title{Resumé title (optional)}               % optional, remove the line if not wanted
\address{Calle Las Pozas 118}{28200 S. L. de El Escorial}    % optional, remove the line if not wanted
\mobile{646636486}                    % optional, remove the line if not wanted
%\phone{phone (optional)}                      % optional, remove the line if not wanted
%\fax{fax (optional)}                          % optional, remove the line if not wanted
\email{fpereda@gmail.com}                      % optional, remove the line if not wanted
%\homepage{homepage (optional)}                % optional, remove the line if not wanted
\extrainfo{Fecha de nacimiento: 29-11-1985} % optional, remove the line if not wanted
\photo[1in]{fjpCV}
%\photo[64pt]{picture}                         % '64pt' is the height the picture must be resized to and 'picture' is the name of the picture file; optional, remove the line if not wanted
%\quote{Some quote (optional)}                 % optional, remove the line if not wanted

% to show numerical labels in the bibliography; only useful if you make citations in your resume
\makeatletter
\renewcommand*{\bibliographyitemlabel}{\@biblabel{\arabic{enumiv}}}
\makeatother

% bibliography with mutiple entries
%\usepackage{multibib}
%\newcites{book,misc}{{Books},{Others}}

%\nopagenumbers{}                             % uncomment to suppress automatic page numbering for CVs longer than one page
%----------------------------------------------------------------------------------
%            content
%----------------------------------------------------------------------------------
\begin{document}
\maketitle

\section{Formación Académica}
\cventry{9/2010 -- 7/2011}{Máster en Investigación en Informática}{Universidad Complutense de Madrid}{}{}{
	\small
	Nota media: 9.63. \\
	Trabajo Fin de Máster: \emph{Guiado, navegación y control de una flota de barcos autónomos}, \\
	Matrícula de Honor.}
\cventry{9/2003 -- 7/2010}{Ingeniero en Informática}{Universidad Carlos III de Madrid}{}{}{
	\small
	Nota media: 7.2. \\
	Proyecto Fin de Carrera: \emph{Sistema de telemetría y control de un barco autónomo}, \\
	Matrícula de Honor.}

\section{Experiencia}
\cventry{desde 12/2010}{Ingeniero}{UCM / \emph{European Defense Agency} / Ministerio de Defensa}{}{}{
Programa NECSAVE.\newline{}%
Estudio e investigación en temas de control de flotas de barcos autónomos:%
\begin{itemize}%
	\item Control no lineal y modelos de maniobra.
	\item Estimación y fusión multisensorial (Kalman).
	\item Sistemas de guiado automático.
\end{itemize}}

\cventry{11/2009 -- 12/2010}{Ingeniero Software}{UCM / Navantia / Ministerio de Defensa}{}{}{
Programa SIRAMICOR.\newline{}%
Diseño e implementación de un simulador de barcos autónomos para desminado automático y de una plataforma experimental.\newline{}%
Trabajos realizados:%
\begin{itemize}%
	\item Implementación y simulación de modelos físicos de barcos.
	\item Diseño e implementación de control automático, tanto en simulación como experimental.
\end{itemize}}

\cventry{11/2008 -- 11/2009}{Responsable de Software}{UCM / \emph{European Space Agency}}{}{}{Diseño e implementación de una estación de tierra y el firmware de un autopiloto para un UAV experimental.\newline{}%
Proyecto SpaceFish, programa RexusBexus.\newline{}%
Trabajos realizados:%
\begin{itemize}%
	\item Desarrollo de los drivers de bajo nivel de los distintos sensores del autopiloto.
	\item Implementación de un sistema de registro de datos en una tarjeta SD.
	\item Desarrollo de un protocolo de comunicación y drivers de radio.
	\item Desarrollo de una estación de tierra para el UAV.
\end{itemize}}

\section{Idiomas}
\cvlanguage{Español}{Lengua materna}{}
\cvlanguage{Inglés}{Nivel alto}{Tanto hablado como escrito. Capaz de mantener una discusión o conversación sin problemas. Escritos varios artículos científicos.}

\section{Competencias informáticas}
\cvline{Lenguajes de programación}{\small C, C++: Utilizados para el desarrollo de distinto software: firmware para el autopiloto, simulador de vehículos autónomos, estación de tierra, ...}
\cvline{Software científico}{\small Matlab: Utilizado para prototipado de algoritmos y para visualización de datos}
\cvline{Sistemas Operativos}{\small Conocimientos extensos de Unix / Linux. Participado activamente en el desarrollo de una distribución de Linux (http://www.gentoo.org).}

\section{Otros datos de interés}
\cvlistitem{Permiso de conducir B.}
\cvlistitem{Disponibilidad para viajar.}
\cvlistitem{Gran interés por el modelado, simulación, estimación y control de vehículos autónomos. Con distintas publicaciones científicas en congresos y un artículo en la revista \emph{IEEE Transactions on Industrial Electronics.}}

% Publications from a BibTeX file without multibib\renewcommand*{\bibliographyitemlabel}{\@biblabel{\arabic{enumiv}}}% for BibTeX numerical labels
\nocite{*}
\bibliographystyle{plain}
% \bibliography{publications}       % 'publications' is the name of a BibTeX file

% Publications from a BibTeX file using the multibib package
%\section{Publications}
%\nocitebook{book1,book2}
%\bibliographystylebook{plain}
%\bibliographybook{publications}   % 'publications' is the name of a BibTeX file
%\nocitemisc{misc1,misc2,misc3}
%\bibliographystylemisc{plain}
%\bibliographymisc{publications}   % 'publications' is the name of a BibTeX file
\end{document}


%% end of file `template_en.tex'.
